\chapter{Определения}

Если объем выборки достаточно велик ($n > 50$), то элементы выборки группируются в так называемый статистический ряд. Для этого отрезок\newline $J = [x_{(1)}, x_{(n)}]$ разбивают на $m$ равновеликих промежутков.\newline

$$ J_i = [x_{(1)} + (i - 1) \cdot \Delta, x_{(1)} + i \cdot \Delta), i = \overline{1, m - 1} $$

$$ J_{m} = [x_{(1)} + (m - 1) \cdot \Delta, x_{(n)}], $$

где:

\begin{itemize}[label = ---]
    \item $ \Delta = \frac{|J|}{m} = \frac{x_{(n)} - x_{(1)}}{m}, $
    \item $m = \lfloor \log_2 n \rfloor + 2$
    \item $x_{(1)} = min(\vec x)$
    \item $x_{(n)} = max(\vec x)$
\end{itemize}

\section{Интервальный статистический ряд}

\textbf{Определение:} Интервальным статистическим рядом называют таблицу, где $n_i$ --- число элементов выборки $\vec x$, попавших в $J_i$, $i = \overline{1, m}$.

\begin{table}[htb]
    \centering
    \begin{tabular}{|c|c|c|c|c|}
        \hline
        $J_1$ & $\cdots$ & $J_i$ & $\cdots$ & $J_m$ \\
        \hline
        $n_1$ & $\cdots$ & $n_i$ & $\cdots$ & $n_m$ \\
        \hline
    \end{tabular}
\end{table}

\clearpage

\section{Эмпирическая плотность}

Пусть для данной выборки $\vec x = (x_1,\ \ldots,\ x_n)$ построен интервальный статистический ряд $(J_i, n_i)$, $i = \overline{1; m}$

\textbf{Определение:} Эмпирической плотностью, отвечающей выборке $\vec x$, называется функция

$$ 
f_n(x) = 
\begin{cases}
    \frac{n_i}{n \cdot \Delta}, \text{если}\  x \in J_i \\
    0, \text{иначе} \\
\end{cases}
$$

\section{Гистограмма}

\textbf{Определение:} Гистограммой называется график эмпирической функции плотности

\section{Эмпирическая функция распределения}

Пусть $\vec x = (x_1, ..., x_n)$ --- выборка из генеральной совокупности $X$.

Обозначим $l(t, \vec x)$ --- число элементов выборки $\vec x$, которые меньше $t$.

\textbf{Определение:} Эмпирической функцией распределения, отвечающей выборке $\vec x$, называют отображение

$$ F_n: {\mathbb{R}} \to {\mathbb{R}}, $$

определенное правилом:

$$ F_n(t) = \frac{l(t, \vec x)}{n}. $$
