\chapter{Определения}

\section*{Определение $\gamma$-доверительного интервала для значения параметра распределения случайной величины}

Пусть:

\begin{itemize}[label=]
    \item $X$ --- случайная величина, закон которой известен с точностью до неизвестного $\theta$.
    \item $\vec X = X_1,...,X_n$ - случайная выборка из генеральной совокупности $X$.
\end{itemize}

\textbf{Опр} Интервальной оценкой параметра $\theta$ уровня $\gamma \in (0, 1)$ ($\gamma$ - интервальной оценкой) называется пара статистик:

\begin{center}
	$\underline{\theta} (\vec X)$ и $\overline{\theta} (\vec X)$ 
\end{center}
    таких, что
\begin{center}
    $P \{\theta \in (\underline{\theta} (\vec X), \ \overline{\theta} (\vec X))\} = \gamma$.
\end{center}

$\underline{\theta} (\vec X)$ и $\overline{\theta} (\vec X)$ называют нижней и верхней границами интервальной оценки соответственно.

\textbf{Опр} $\gamma$ - доверительным интервалом (доверительным интервалом уровня $\gamma$) для параметра $\theta$ называют реализацию интервальной оценки уровня $\gamma$ для этого параметра, то есть интервал:

\begin{center}
	$(\underline{\theta} (\vec x), \ \overline{\theta} (\vec x))$,
\end{center}

где $\vec x$ --- выборка из генеральной совокупности $X$.